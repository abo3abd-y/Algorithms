\documentclass[11pt]{article}
\usepackage{amsfonts, amssymb, amsmath}
\usepackage{float}
\parindent 0px
% we do not indent
\pagestyle{empty}

\begin{document}

The distributive property states that $a(b+c) = ab + ac$, for all $a,b,c \in \mathbb{R}$\\[6pt]
The equivalence class of $a$ is $[a]$.\\[6pt]
The set $A$ is defined to be $\{1,2,3\}$\\[6pt]
% if you want the curly brackets to show up, use \ before {} so \{ and \}
The movie ticket costs $\$11.50$
% same thing with dollar signs

$$2\left(\frac{1}{x^2 - 1}\right)$$
$$2\left[\frac{1}{x^2 - 1}\right]$$
$$2\left\{     \frac{1}{x^2 - 1}\right\}$$
$$2\left \langle     \frac{1}{x^2 - 1}      \right \rangle$$
$$2\left |     \frac{1}{x^2 - 1}      \right |$$
% as you can see, paranthesis looks trash because they dont fit to what is inside of them.
% which is why you use \left( for the left one and \right) for the right one



$$\left.\frac{dy}{dx}\right|_{x=1}$$
% whats happening here? we only want the right, and not the left. But, in latex both have to be here. So, you can put a dot in the left so it does not appear.




$$\left(\frac{1}{1+\left(\frac{1}{1 + x}\right)}\right)$$


Tables:\\

\begin{tabular}{|c||c|c|c|c|c|} % use c for centering and in here we are basically determining how many columns we'll have
    % the vertical bar above is basically the line that seperates columns
    \hline % this is just a line
    $x$ & 1 & 2 & 3 & 4 & 5 \\ % use \\ to indicate that we are done with the first row
    \hline
    $f(x)$ & 10 & 11 & 12 & 13 & 14
\end{tabular}


\vspace{1cm}

\begin{table}[H] % we are telling latex that we want the table in the same area as typing, not to the top of page
\centering % centers the table with the caption
\def\arraystretch{1.5} % this add spacing in the number and the line
\begin{tabular}{|c||c|c|c|c|c|}
    \hline
    $x$ & 1 & 2 & 3 & 4 & 5 \\ % use \\ to indicate that we are done with the first row
    \hline
    $f(x)$ & $\frac{1}{2}$  & 11 & 12 & 13 & 14
\end{tabular}
\caption{These values represent the function $f(x)$}
% we put a caption below the table
\end{table}



% you can do space in math mode by doing the following: \,

Arrays:
\begin{align} % align is already in math mode
    5x^2-9=x+3\\
    5x^2-x-12=0
\end{align}

% putting the &= means that the equal signs are aligned
\begin{align*} % align is already in math mode
    5x^2-9&=x+3\\
    5x^2-x-12&=0\\
    &=12+x-5x^2
\end{align*}
% the atrick removes the numbers






\end{document}